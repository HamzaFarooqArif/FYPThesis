\documentclass[12pt]{article}
\usepackage[english]{babel}
\usepackage[utf8]{inputenc}
\usepackage{helvet}
\usepackage{graphicx}
\usepackage{longtable}
\usepackage{multirow}
\usepackage[unicode=true, hidelinks]{hyperref}
\setlength{\parindent}{0pt}
\setlength{\parskip}{6pt plus 2pt minus 1pt}
\setcounter{secnumdepth}{4}
\usepackage{array}
\usepackage{float}

\newcommand{\plogo}{\fbox{$\mathcal{PL}$}} % Generic dummy publisher logo

\usepackage[utf8]{inputenc} % Required for inputting international characters
\usepackage[T1]{fontenc} % Output font encoding for international characters
%\usepackage{fouriernc} % Use the New Century Schoolbook font

\renewcommand{\listtablename}{Tables}
\graphicspath{ {./images/} }

%\title{
%\textbf{
%{\fontsize{30}{104}\selectfont Software Specification Requirement}
%\\* For
%\\* {\fontsize{30}{104}\selectfont Digital Board Marker}
%}
%}
%\date{\vspace{-5ex}}


\begin{document}

%\maketitle
%\begin{flushright}
%\textbf{Prepared by:}
%\vspace{0.5cm}
%\\* \textbf{Muhammad Haris Khan, Hamza Farooq, Ayesha Atif, Komal Shehzadi}
%\vspace{0.5cm}
%\\* \textbf{Department of Computer Science and Engineering, UET Lahore.}
%\end{flushright}

\begin{titlepage} % Suppresses headers and footers on the title page

	\centering % Centre everything on the title page
	
	\scshape % Use small caps for all text on the title page
	
	\vspace*{\baselineskip} % White space at the top of the page
	
	%------------------------------------------------
	%	Title
	%------------------------------------------------
	
	\rule{\textwidth}{1.6pt}\vspace*{-\baselineskip}\vspace*{2pt} % Thick horizontal rule
	\rule{\textwidth}{0.4pt} % Thin horizontal rule
	
	\vspace{0.75\baselineskip} % Whitespace above the title
	
	{\LARGE Research Paper Summary\\ For\\ Digital Board Marker\\} % Title
	
	\vspace{0.75\baselineskip} % Whitespace below the title
	
	\rule{\textwidth}{0.4pt}\vspace*{-\baselineskip}\vspace{3.2pt} % Thin horizontal rule
	\rule{\textwidth}{1.6pt} % Thick horizontal rule
	
	\vspace{2\baselineskip} % Whitespace after the title block
	
	%------------------------------------------------
	%	Subtitle
	%------------------------------------------------
	

%	Version 3.0
	
%	\vspace*{3\baselineskip} % Whitespace under the subtitle
	
	%------------------------------------------------
	%	Editor(s)
	%------------------------------------------------
	
%	Prepared By
%	
%	\vspace{0.5\baselineskip} % Whitespace before the editors
	
%	{ Muhammad Haris Khan(2016-CS-105) \\ Hamza Farooq(2016-CS-122) \\ Ayesha Atif(2016-CS-152) \\ Komal Shehzadi(2016-CS-178)} % Editor list
	
%	\vspace{0.5\baselineskip} % Whitespace below the editor list
	
%	\textit{University of Engineering and Technology \\ Lahore} % Editor affiliation
	
%	\vfill % Whitespace between editor names and publisher logo
	
	%------------------------------------------------
	%	Publisher
	%------------------------------------------------
	
	
	
	\vspace{0.3\baselineskip} % Whitespace under the publisher logo
	
	\textit{Copyright \textsuperscript{\textcopyright} 2019 all rights reserved} % Publication year
	


\end{titlepage}


\section{The impact of flipped classrooms on student achievement in engineering education: A meta-analysis of 10 years of research}
A flipped classroom is an instructional strategy and a type of blended learning that reverses the traditional learning environment by delivering instructional content, often online, outside of the classroom. It moves activities, including those that may have traditionally been considered homework, into the classroom.\\
The meta-analysis comparing these 29 traditional flipped interventions in relation to student achievement showed an overall significant effect in favor of the flipped classroom over traditional lecturing (Hedges' g =0.289, 95\% CI [0.165, 0.414], p< .001). A moderator analysis showed that the effect of the flipped classroom was further enhanced when instructors offered a brief review at the start of face-to-face classes.\\
They calculated students learning rate by means of hedges' g which tells whether the impact of flipped classroom is positive or negative and the impact of flipped classrooms was mostly positive the learning rate of students increased and passing rate of class also increased. That means flipped classroom was much more good than traditional classrooms.\\
They used number of samples to make sure the conclusion and in majority of samples result was same that flipped classroom are an innovation in educational institutions.


\section{Effect of Prior Knowledge on Attitudes, Behavior, and Learning Performance in Video Lecture Viewing}
This paper shows how a student behaves in general if lecture related material or videos are provided first. They categorized students in two categories the one with low prior knowledge LK and the one with high prior knowledge HK then they used these samples to examine the behavior of students in each sample to check with prior knowledge how many times students does what actions like start over rewind a video after the lecture so that they can see how much interest students are showing and how these videos are helping them in their studies and students with high prior knowledge had more potential and passed their exams with more marks then LK.\\
The purpose of this analysis was to answer the following questions:\\
1. Does prior knowledge affect students’ engagement level of viewing video lectures?\\
2. Does prior knowledge affect students’ strategies used for viewing video lectures?\\
3. Does prior knowledge affect students’ learning performance when learning from video lectures?\\
4. Does prior knowledge affect learners’ attitudes toward video lectures?\\


Effect was positive for all the students with high prior knowledge they have to do less effort.


\section{Improving effectiveness of learners’ review of video lectures by using an attention-based video lecture review mechanism based on brainwave signals}
This system was made for the benefit of students so that they can learn everything on the basis of their attention. Video lectures are common but this system records the patches of video based on the attention of students on brain signals that if they missed any part which was important they can watch it later and know that they missed this thing.




\section{Recorded Lectures as a Source of Cognitive Off-loading}
This study was done for the students of pharmacy to see their response and performance if they get recorded lectures or not study shows that students have more storage space like a storage device when we tell them they can have the recorded lectures so they can focus on other things which are more productive and it will increase their performance because now they have a medium for later use so that they can use their brain for other important things study shows how human minds works as a storage space if we tell them that they will not get recorded lectures they will save their storage space to absorb the lecture and fully remind so that later on they dont get any difficulty but in spite of helping them it will stop them from growing.

\section{Anchoring interactive points of interest on web-based instructional video: effects on students’ interaction behavior and perceived experience}  
This research shows how a marked timeline of a video lecture affects the behavior of students they took two groups of students as sample and showed one group conventional video of lectures and other group the marked timeline video and the experiment showed how the behavior of students who were shown a marked timeline with topics or keynotes was different they clicked less on timeline and their pattern of clicking the timeline was almost same and later on they performed well on that test or quiz held from that video because of the interactive timeline the behavior of students changes and the feel more open to learn and explore.

\section{Effects on learning of time spent by university students attending lectures and/or watching online videos}
This study shows how online lectures plays a good role in the learning of students they provided online videos to students with lectures and recorded their time spent on lecture and video lectures and study showed students uses online videos as a substitute of lectures and it is a positive thing to learn those who spent time in lecture have same learning to those who spent equal time on video lectures.


\section{Guided notes for university mathematics and their impact on students’ note-taking behaviour}
This research focuses on the behavior of students while taking notes typically students note everything written on board and a very few of them write on their own to find links between lecture and their learning and this thing is good for the growth of any student but a few does it because a student brain thinks what a teacher is writing on the board is most important and after that what he is saying is important very few thinks what they are learning and thats whats needed in notes taking for a productive notes taking of students.


\section{Student perception of helpfulness of facilitation strategies that enhance instructor presence, connectedness, engagement and learning in online courses}
This study shows how some strategies work for the students interaction and interest in an online course teacher uses different strategies to engage students in an online course because it is so easy for students to just let it go they can just quit it so all the teachers around the globe uses some strategies to develop the interest of student and in this study they examined 12 strategies to see the behavior of students.\\
Instructor presence\\
Instructor Connectedness\\
Engagement \\
Learning\\
All 12 strategies constructs around these four factors. and these strategies are:\\
Video based instructor introduction (e.g., Voicethread, Animoto, Camtasia)\\
Video based course orientation (e.g., recording using Camtasia, screencast o matic)\\
Able to contact the instructor in multiple ways (contact the instructor forum, email, phone, virtual office hours)\\
Instructors timely response to questions (e.g., within 24 to 48 h) via forums, email\\
Instructors weekly announcements to the class (e.g. every Monday via announcement forum, email)\\
Instructor created content in the form of short videos/multimedia (e.g., Camtasia, articulate modules)\\
Instructor being present in the discussion forums (e.g., refers to students by name, responds to students posts)\\
Instructor providing timely feedback on assignments/projects (e.g., within 7days)\\
Instructor providing feedback using various modalities (e.g., text, audio, video, and visuals) on assignments/projects\\
Instructors personal response to student reflections (e.g., via journals to questions on benefits/challenges\\
Instructors use of various features in synchronous sessions to interact with students (e.g., polls, emoticons, whiteboard, text, or audio and video chat)\\
Interactive visual syllabi of the course (e.g., includes visual of the instructor and other interactive components)\\


\section{Design of a learning-centered online environment: a cognitive apprenticeship approach}
The design of online course materials is rarely informed by learning theories or their pedagogical implications. The goal of this research was to develop, implement and assess a virtual learning environment (VLE), SOFIAA, which was designed using the cognitive apprenticeship model (CAM), a pedagogical model based on learning-centered theory. We present an instructional design case study that reveals the steps taken to improve student performance in a master’s level blended learning course on program evaluation. The case study documents four phases of improving on-line instruction in program evaluation, starting with Online Course Materials (OCM) that contained resources and information required to complete team field projects. In phase 1, quantitative analyses revealed that there was improvement of student test scores using the OCM, however, qualitative analyses of think-aloud sessions found that students failed to attain key course objectives. In phase 2, a team of experts reviewed the materials and suggested ways to improve opportunities for student learning. In phase 3, a (VLE) was designed based on the results of phase 2 using a reconceptualization of CAM as a design model. In phase 4, the VLE was validated using experts’ appraisal of content and presentation, and student achievement, which indicated that use of the VLE led to significant improvement in learning over use of OCM. The design process is discussed in terms of a reconceptual- ization of CAM as a general strategy for instructional design that can be used to improve both the content and quality of online course materials.

\section{Note This: How to Improve Student Note Taking}
Students are incomplete note takers who routinely record just one third of a lesson’s important information in their notes. This is unfortunate, because the number of lesson points recorded in notes is positively correlated with student achievement. Moreover, both the activity of recording notes and the subsequent review of notes are advantageous. The authors offer instructors a menu of research-based advice for bolstering student note taking: provide complete notes, provide partial notes, provide note- taking cues, re-present the lesson, provide pauses and revision opportunities, control laptop usage, control “cyber slacking,” use PowerPoint slides effectively, and teach note-taking skills. They also suggest ways to help students transform their notes during the note-review process and SOAR (select, organize, associate, and regulate) to success.
This study shows how notes taking helps student and how student are not good with it but we have to do something to improve it so their are some things or remedies we can do to improve it because notes are basically most essential thing for a student to get through the exams.

\section{Did you take “good” notes?: On methods for evaluating student notetaking performance}
This paper shows how to check whether a student is taking good notes because taking notes is a good thing obviously but only if you are taking it right otherwise they are going to mess with your head later when you read them and instead of being productive it will destroy your concepts this paper also shows how to write good notes and how a teacher can judge that a student is taking notes correctly based on some keynotes.

\section{Engagement Matters: Student Perceptions on the Importance of Engagement Strategies in the Online Learning Environment}
Student engagement increases student satisfaction, enhances student motivation to learn, reduces the sense of isolation, and improves student performance in online courses. This survey-based research study examines student perception on various engagement strategies used in online courses based on Moore’s interaction framework. One hundred and fifty-five students completed a 38-item survey on learner-to-learner, learner-to-instructor, and learner-to-content engagement strategies. Learner-to-instructor engagement strategies seemed to be most valued among the three categories. Icebreaker/introduction discussions and working collaboratively using online communication tools were rated the most beneficial engagement strategies in the learner-to-learner category, whereas sending regular announcements or email reminders and providing grading rubrics for all assignments were rated most beneficial in learner-to-instructor category. In the learner-content category, students mentioned working on real-world projects and having discussions with structured or guiding questions were the most beneficial. This study also analyzed the effect of age, gender, and years of online learning experience differences on students’ perception of engagement strategies. The results of the study have implications for online instructors, instructional designers, and administrators who wish to enhance engagement in the online courses.
Basically, this paper shows how engagement matters in an online course because online courses can be a complete waste if we dont do it properly and study properly in it includes engagement of the student that he can follow the course as the instructor is teaching and what strategies we should use to engage a student in an online course. 


\end{document}
