% Chapter 1

\chapter{Introduction} % Write in your own chapter title
\label{Chapter1}
\lhead{Chapter 1. \emph{Project Report/Thesis Structure}} % Write in your own chapter title to set the page header
\section{Background and Overview}
Digital Board Marker is a size efficient, bandwidth saving lecture recording system. It can record lecture, providing automated google search of handwritten words. It provides on the spot wiki. Lecture text notes can be generated automatically. Lecture can be named and divided into topics and subtopics automatically. According to a survey, 94\% students go for online help of recently attended lectures because they can’t fully grab the concepts. Recorded lectures as video format require so much internet bandwidth to play. In most cases, large sized videos are difficult to handle or download. Because students mostly don’t have huge amount of extra space available especially for the CSE students, as they already use bulky software and also students don’t have large amount of bandwidth of internet available.
Digital Board Marker is an innovation and need of the hour because it is solving the basic problem of all students in society because it saves their time to note the lecture and they can concentrate on the topic completely. As most of the student lost their concentration while writing the detailed lecture. Moreover, sometimes we skip important part of our lecture in the effort of writing whole details on our notebook. So, Digital Board Marker is leading-edge to solve all these Issues. It has some potential for having positive effect on student learning.

\section{Motivation}
The motivation and purpose to do this project is to minimize the use of resources that are used in lecture systems now a days working in all over the world i.e. video lecture recording and streaming through internet. The first motivation is to deal with the large amount of storage that normally video lectures take. This system is not based on video recording but on recording the writing on the board with marker. It will record the position of the marker as the coordinates of board where marker touches and store it in the text file (which will later be converted and played like a video). This will take minimum amount of database storage to store this kind of data on a website. The second motivation to do this project is to use less internet resources for accessing the lectures. Normally the video lectures of different institutes worldwide are very large and to download those on the system through internet requires large amount of resources which are normally difficult for students to get and to download it in high quality even more resources are required. The lectures for recording are very low in memory as compared to normal video recording and will require very minimum resources to download on the system. The third motivation is for example a power failure occurred during the lecture and you cannot clearly see the board but teacher is still writing and erases the board after some time, this may result in not getting proper notes or missing the important point of lecture. Moreover students can get benefit by seeing the lecture again and again if they missed any concept or if they were absent minded or not attending lecture. These few are the reasons which motivated us to do this project.

\bigskip
\section{Objectives of the Project}
\bigskip
\subsection{Industry Objectives}
In industry most of the time it is hard to choose areas for work which have low bandwidth internet and let's suppose you are playing a lagging call of duty run-through and your stream is buffering and stopping because of low bandwidth it's like you are losing because of this or you are presenting something which is improved work of someone else and It requires high quality fast internet to present it but it's not guaranteed.
\par In some places people try to reduce the cost of these things as much as possible but not having proper interface is the main reason of failure so we can cop up with this issue by this new system we are introducing.

\begin{itemize}

\item System will reduce internet bandwidth usage which 
will lead to progress in industry.
\item It must minimize the storage issue which can increase working efficiency of industry.
\item It will reduce the cost of internet and cost of storage and will help the industry in fast growing world of today.
\item Main aim of the project is to provide ease and best performance than most previous ones and will eventually lead to progress in industrial field.

\end{itemize}

\subsection{Research Objectives}
In the development of digital board marker, computer vision is used and computer vision is most vital in the field of research. Computer vision plays a great role in research work. So by improving the uses of computer vision in future work its vast area for research work.
Research objective of the system is to go through all the recent research work done in system's development fields and then on its basis, developing a system which is storage and bandwidth efficient.

\begin{itemize}

\item Project research is related to find position and orientation of marker precisely and accurately.
\item The high level research part is finding the position and then syncing it with the audio data to play like a video.
\item Research must be deep so that researchers must be able to discover new and improved techniques to reduce storage issues.
\item Research should be able to help future work in detection of ball in any sort environment without assumptions and with more accuracy.

\end{itemize}

\subsection{Academic Objectives}
Digital board marker mainly cover academic area the main purpose is to provide each and every student all the lectures with better quality and less bandwidth because in Pakistan we students face this issue the most, as we know it cannot be resolved in near future we have to work something out for this issue and that's where this system will work it will provide an interface to all the students which have all the lectures of their respective subjects from their respective teachers which can be streamed online and downloaded for offline to play later on at very low bandwidth. It will provide all the assignment related material and lectures at same platform to students. It is the new revolution in the academic field.

\begin{itemize}

\item The main academic objective of the developers is to learn major computer science field i.e. computer vision.

\item On basis of computer fields used in project developers must be able to use this knowledge to improve in this field.

\item Developers must complete all the work before respective deadlines. so by working in a professional way project will be at its best.

\item Developers must be able to risk the change management in their projects, as while doing the projects, developers might face different kinds of situation and their decision making plays and important role in leading them to success.

\end{itemize}


\section{Problem Statement}
To make a storage and bandwidth efficient system with a lecture player and learning management system for the students and the educational institutes.
\bigskip

\section{Scope of the Project}
Digital board marker mainly cover academic area the main purpose is to provide each and every student all the lectures with better quality and less bandwidth because in Pakistan we students face this issue the most, as we know it cannot be resolved in near future we have to work something out for this issue and that’s where this system will work it will provide an interface to all the students which have all the lectures of their respective subjects from their respective teachers which can be streamed online and downloaded for offline to play later on at very low bandwidth. It will provide all the assignment related material and lectures at same platform to students. It is the new revolution in the academic field. Although it covers industry and researches as well.
























