% Chapter 2

\chapter{Literature Review} % Write in your own chapter title
\label{Chapter2}
\lhead{Chapter 2. \emph{Literature Review}} % Write in your own chapter title to set the page header

\section{Literature Review}
This paper presents a technique that aimed to accomplish an efficient balance between video compression using H.265 protocol and retention of 8K resolution. The study implements multi-level of optimization in the encoding process using H.265 where JPEG2000 standards play a crucial role. The study also applies a novel concept of orthogonal projection that manages pixels metadata required in every frame transition followed by motion compensation. By using multiple file formats of 30 video datasets, the outcome of the study is found to be accomplishing approximately 49\% of enhancement in data quality and around 59\% of improvement in video compression in comparison to the existing techniques of HEVC-based video compression.\cite{Murthy2016}


Web-based lecture technologies are being used increasingly in higher education. One widely-used method is the recording of lectures delivered during face-to-face teaching of on-campus courses. The recordings are subsequently made available to students on-line and have been variously referred to as lecture capture, video podcasts, and Lectopia. We examined the literature on lecture recordings for on-campus courses from the perspective of students, lecturers, and the institution. Literature was drawn from major international electronic databases of Elsevier ScienceDirect, PsycInfo, SAGE Journals, SpringerLink, ERIC and Google Scholar. Searches were conducted using key terms of lecture capture, podcasts, vodcasts, video podcasts, video streaming, screencast, webcasts, and online video. The reference sections of each article were also searched and a citation search was conducted. Institutions receive pressure from a range of sources to implement web-based technologies, including from students and financial imperatives, but the selection of appropriate technologies must reflect the vision the institution holds. Students are positive about the availability of lecture recordings. They make significant use of the recordings, and the recordings have some demonstrated benefits to student learning outcomes. Lecturers recognise the benefits of lecture recordings for students and themselves, but also perceive several potential disadvantages, such as its negative effect on attendance and engagement, and restricting the style and structure of lectures. It is concluded that the positives of lecture recordings outweigh the negatives and its continued use in higher education is recommended. However, further research is needed to evaluate lecture recordings in different contexts and to develop approaches that enhance its effectiveness.\cite{OCallaghan2017}

The flipped classroom has become more widely used in engineering education. However, a systematic and quantitative assessment of its achievement outcomes has not been conducted to date. Purpose: To address this gap, we examined the findings from comparative articles published between 2008 and 2017 through a meta-analysis to summarize the overall effects of the flipped classroom on student achievement in engineering education. We searched and analyzed journal and conference publications on flipped classroom studies in engineering education in K-12 and higher education contexts. Twenty-nine comparative interventions were included in a meta-analysis involving 2,590 students exposed to flipped classroom and 2,739 students exposed to traditional lectures. A content analysis was also conducted to determine how the flipped engineering classroom benefits student learning. Conclusions: The meta-analysis comparing these 29 traditional flipped interventions in relation to student achievement showed an overall significant effect in favor of the flipped classroom over traditional lecturing (Hedges' g = 0.289, 95\% CI [0.165, 0.414], p <.001). A moderator analysis showed that the effect of the flipped classroom was further enhanced when instructors offered a brief review at the start of face-to-face classes. Our qualitative findings suggest that self-paced learning and more problem-solving activities were the two most frequently reported benefits that promoted student learning. Based on quantitative and qualitative support, several implications are identified for future practice, such as offering a brief in-class review of preclass materials. Some recommendations for future research are also provided.\cite{Lo2019}


Videos have enhanced the value of teaching and learning, particularly in tertiary education. Recent studies have investigated students' attitudes toward video lectures for educational purposes; however, the relationship between students' attitudes and different usage patterns such as platforms used, video duration, watching period and students' experience, is yet to be explored. To investigate potential attitudinal differences among the diverse video lectures usage patterns, the present study incorporates responses from 40 students who participated in a video-assisted software engineering course. Our results suggest that usage patterns affect students' attitudes to video lectures as a learning tool. The overall outcomes are expected to promote theoretical development of students' attitudes, video-platform design principles, and better and more efficient use of video lectures.\cite{Giannakos2016}



The literature is mixed as to whether the addition of lecture capture technologies provide for better student success. In this work, we consider not just the broad effect of lecture capture technology on academic achievement between cohorts, but whether this effect is related to patterns of viewership among learners. At the centre of our interest is determining whether there are strategies learners take in their reviewing of content week-to-week that may result in better achievement. To investigate this, we describe a method for modelling learners based on their interactions with lecture capture systems. Unlike investigations done by others, our models emerge from the activities of the learners themselves, and are based on the results of applying unsupervised machine learning (clustering) techniques to student viewership data. These models describe five different classifications of learner interactions, and we show that one of these is positively correlated with academic achievement. We further validate our results through repeated experimentation, and describe how such models might be used by early-alert systems.\cite{Brooks2014}


Instructors use various strategies to facilitate learning and actively engage students in online courses. In this study, we examine student perception on the helpfulness of the twelve different facilitation strategies used by instructors on establishing instructor presence, instructor connection, engagement and learning. One hundred and eighty eight graduate students taking online courses in Fall 2016 semester in US higher education institutions responded to the survey. Among the 12 facilitation strategies, instructors' timely response to questions and instructors' timely feedback on assignments/projects were rated the highest in all four constructs (instructor presence, instructor connection, engagement and learning). Interactive visual syllabi of the course was rated the lowest, and video based introduction and instructors' use of synchronous sessions to interact were rated lowest among two of the four constructs. Descriptive statistics for each of the construct (instructor presence, instructor connection, engagement and learning) by gender, status, and major of study are presented. Confirmative factor analysis of the data provided aspects of construct validity of the survey. Analysis of variance failed to detect differences between gender and discipline (education major versus non-education major) on all four constructs measured. However, undergraduate students rated significantly lower on engagement and learning in comparison to post-doctoral and other post graduate students.\cite{Martin2018}


This paper reports findings from a case study of the impact that teaching using guided notes has on university mathematics students’ note-taking behaviour. Whereas previous research indicates that students do not appreciate the importance of lecturers’ non-written comments and record in their notes only what is written on the board when taught with the traditional chalk and talk method, some students in our study recorded the non-written comments as well as some of their own links between sections of the lecture. We did not, however, find students’ attitude towards those comments to be different from what previous research found. We conclude that guided notes can be an appropriate way of teaching university mathematics but on their own cannot make the pedagogical intentions of the lecturer clearer to the students. We also found that the educational environment plays a big part for all aspects of student learning, including decisions related to note-taking during lectures.\cite{Iannone2019}


Online video lectures are widely used in e-learning environments. They provide several advantages for students such as preparing for class and controlling their learning pace. However, essential features of videos, such as transient information and learner control, can also increase learners’ cognitive load and disorientation, particularly for learners with low prior knowledge. This study analyzed data collected from a questionnaire, students’ examination and homework scores, and system logs to examine the effects of prior knowledge on the engagement level, frequency of viewing strategies used, attitudes, and learning performance of students who watched video lectures. The results showed that the students demonstrated the same engagement levels of watching video lectures, regardless of whether they had high or low prior knowledge. However, high prior knowledge learners used a higher frequency of viewing strategies, had a more positive attitude toward watching the video lectures, and exhibited higher learning performance than the low prior knowledge learners did. These results are discussed in this article, and several suggestions for personalized prior knowledge support are proposed.\cite{Li2019}