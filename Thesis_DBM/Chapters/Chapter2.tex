% Chapter 2

\chapter{Literature Review} % Write in your own chapter title
\label{Chapter2}
\lhead{Chapter 2. \emph{Literature Review}} % Write in your own chapter title to set the page header

\section{Literature Review}
This paper presents a technique that aimed to accomplish an efficient balance between video compression using H.265 protocol and retention of 8K resolution. The study implements multi-level of optimization in the encoding process using H.265 where JPEG2000 standards play a crucial role. The study also applies a novel concept of orthogonal projection that manages pixels metadata required in every frame transition followed by motion compensation. By using multiple file formats of 30 video datasets, the outcome of the study is found to be accomplishing approximately 49\% of enhancement in data quality and around 59\% of improvement in video compression in comparison to the existing techniques of HEVC-based video compression.\cite{Murthy2016}


Web-based lecture technologies are being used increasingly in higher education. One widely-used method is the recording of lectures delivered during face-to-face teaching of on-campus courses. The recordings are subsequently made available to students on-line and have been variously referred to as lecture capture, video podcasts, and Lectopia. We examined the literature on lecture recordings for on-campus courses from the perspective of students, lecturers, and the institution. Literature was drawn from major international electronic databases of Elsevier ScienceDirect, PsycInfo, SAGE Journals, SpringerLink, ERIC and Google Scholar. Searches were conducted using key terms of lecture capture, podcasts, vodcasts, video podcasts, video streaming, screencast, webcasts, and online video. The reference sections of each article were also searched and a citation search was conducted. Institutions receive pressure from a range of sources to implement web-based technologies, including from students and financial imperatives, but the selection of appropriate technologies must reflect the vision the institution holds. Students are positive about the availability of lecture recordings. They make significant use of the recordings, and the recordings have some demonstrated benefits to student learning outcomes. Lecturers recognise the benefits of lecture recordings for students and themselves, but also perceive several potential disadvantages, such as its negative effect on attendance and engagement, and restricting the style and structure of lectures. It is concluded that the positives of lecture recordings outweigh the negatives and its continued use in higher education is recommended. However, further research is needed to evaluate lecture recordings in different contexts and to develop approaches that enhance its effectiveness.\cite{OCallaghan2017}

