% Chapter 2

\chapter{Literature Review} % Write in your own chapter title
\label{Chapter2}
\lhead{Chapter 2. \emph{Literature Review}} % Write in your own chapter title to set the page header

\section{Literature Review}
As a student have you ever wanted that if you can take the lecture you missed because of some mishap or the concepts delivered in that lecture was too hard to grasp immediately and you wanted to discuss them later with other people but you could not remember most of it because you cannot jot down each and every thing? One time or another we all have been there. To achieve this goal many techniques were used providing notes or recording lectures and later providing them to all the students but since every thing has its own pros and cons so first we have to make sure that these facilities are either productive or destructive many researches has done to support the idea of use of revolutionized lecture systems different techniques were used to achieve this goal but we will get back to those techniques later. First we have to see the use of these facilities is beneficial or not.
The use of video lectures in educational institution is considered as an essential source because students can't pay complete attention in class whole the time. It is important to provide an alternative as good as possible that's why video lectures pay a huge role. Web-based lecture technologies are being used increasingly in higher education. One widely-used method is the recording of lectures delivered during face-to-face teaching of on-campus courses. The recordings are subsequently made available to students on-line and have been variously referred to as lecture capture, video podcasts, and Lectopia. We examined the literature on lecture recordings for on-campus courses from the perspective of students, lecturers, and the institution. Institutions receive pressure from a range of sources to implement web-based technologies, including from students and financial imperatives, but the selection of appropriate technologies must reflect the vision the institution holds. Students are positive about the availability of lecture recordings. They make significant use of the recordings, and the recordings have some demonstrated benefits to student learning outcomes. Lecturers recognize the benefits of lecture recordings for students and themselves, but also perceive several potential disadvantages, such as its negative effect on attendance and engagement, and restricting the style and structure of lectures. It is concluded that the positives of lecture recordings outweigh the negatives and its continued use in higher education is recommended. However, further research is needed to evaluate lecture recordings in different contexts and to develop approaches that enhance its effectiveness.\cite{OCallaghan2017}\\
Then a research team decided to see the impact of flipped classroom on students and their growth. A flipped classroom is an instructional strategy and a type of blended learning that reverses the traditional learning environment by delivering instructional content, often online, outside of the classroom. It moves activities, including those that may have traditionally been considered homework, into the classroom.\\
The meta-analysis comparing these 29 traditional flipped interventions in relation to student achievement showed an overall significant effect in favor of the flipped classroom over traditional lecturing (Hedges' g = 0.289, 95\% CI [0.165, 0.414], p \textless .001). A moderator analysis showed that the effect of the flipped classroom was further enhanced when instructors offered a brief review at the start of face-to-face classes.\\
They calculated students learning rate by means of hedges' g which tells whether the impact of flipped classroom is positive or negative and the impact of flipped classrooms was mostly positive the learning rate of students increased and passing rate of class also increased. That means flipped classroom was much more good than traditional classrooms. They used number of samples to make sure the conclusion and in majority of samples result was same that flipped classroom are an innovation in educational institutions.\cite{Lo2019}\\
Then we have to see the effects of prior knowledge on students because different approaches can be used to increase the growth rate of students so it can be a positive approach. A research was conducted to analyze these results and a research paper was published as a result of data set in 2019. That paper shows how a student behaves in general if lecture related material or videos are provided first. They categorized students in two categories the one with low prior knowledge LK and the one with high prior knowledge HK then they used these samples to examine the behavior of students in each sample to check with prior knowledge how many times students does what actions like start over rewind a video after the lecture so that they can see how much interest students are showing and how these videos are helping them in their studies and students with high prior knowledge had more potential and passed their exams with more marks then LK.\\
The purpose of this analysis was to answer the following questions:\\
1. Does prior knowledge affect students’ engagement level of viewing video lectures?\\
2. Does prior knowledge affect students’ strategies used for viewing video lectures?\\
3. Does prior knowledge affect students’ learning performance when learning from video lectures?\\
4. Does prior knowledge affect learners’ attitudes toward video lectures?\\
Effect was positive for all the students with high prior knowledge they have to do less effort.\cite{Li2019}\\
Then a system was made to use the video lectures but to make them more effective it was based on the brainwaves of learner. This system was made for the benefit of students so that they can learn everything on the basis of their attention. Video lectures are common but this system records the patches of video based on the attention of students on brain signals that if they missed any part which was important they can watch it later and know that they missed this thing.\cite{Lin2019}
A study was done in 2019 to see the benefit of recorded lectures that they are decreasing the burden off for both students and teachers and how they are helpful for them. This study was done for the students of pharmacy to see their response and performance if they get recorded lectures or not study shows that students have more storage space like a storage device when we tell them they can have the recorded lectures so they can focus on other things which are more productive and it will increase their performance because now they have a medium for later use so that they can use their brain for other important things study shows how human minds works as a storage space if we tell them that they will not get recorded lectures they will save their storage space to absorb the lecture and fully remind so that later on they don't get any difficulty but in spite of helping them it will stop them from growing.\cite{Patel2019}\\
Sometimes customized things are better in educational system because everyone has their own method of teaching and learning for that purpose video lectures are used but what if we change the way of delivery of those lectures by marking the timeline so that students know which part is important for them for that purpose a research was done and published in 2019. That research shows how a marked timeline of a video lecture affects the behavior of students they took two groups of students as sample and showed one group conventional video of lectures and other group the marked timeline video and the experiment showed how the behavior of students who were shown a marked timeline with topics or keynotes was different they clicked less on timeline and their pattern of clicking the timeline was almost same and later on they performed well on that test or quiz held from that video because of the interactive timeline the behavior of students changes and the feel more open to learn and explore.\cite{Pimentel2019}\\
Now if we are using video lectures then we have to see the effects on audience on them. For that purpose a study was conducted on the effects of video lectures on university students and how they spend their time on them. This study shows how online lectures plays a good role in the learning of students they provided online videos to students with lectures and recorded their time spent on lecture and video lectures and study showed students uses online videos as a substitute of lectures and it is a positive thing to learn those who spent time in lecture have same learning to those who spent equal time on video lectures.\cite{Meehan2019}\\
Video lectures was invented because student cannot take notes about each and everything in lecture and they miss things. A research was conducted to see how it can be helpful to take effective notes in classroom. This research focuses on the behavior of students while taking notes typically students note everything written on board and a very few of them write on their own to find links between lecture and their learning and this thing is good for the growth of any student but a few does it because a student brain thinks what a teacher is writing on the board is most important and after that what he is saying is important very few thinks what they are learning and thats whats needed in notes taking for a productive notes taking of students.\cite{Iannone2019}\\
Online lectures have their own cons but they can be recovered by a number of things a research was conducted in 2018 to study the effect of these facilitation in online courses. This study shows how some strategies work for the students interaction and interest in an online course teacher uses different strategies to engage students in an online course because it is so easy for students to just let it go they can just quit it so all the teachers around the globe uses some strategies to develop the interest of student and in this study they examined 13 strategies to see the behavior of students.\\
1. Instructor presence\\
2. Instructor Connectedness\\
3. Engagement \\
4. Learning\\
All 13 strategies constructs around these four factors. and these strategies are:\\
1. Video based instructor introduction (e.g., Voicethread, Animoto, Camtasia)\\
2. Video based course orientation (e.g., recording using Camtasia, screencast o matic)\\
3. Able to contact the instructor in multiple ways (contact the instructor forum, email, phone, 4. virtual office hours)\\
5. Instructors timely response to questions (e.g., within 24 to 48 h) via forums, email\\
6. Instructors weekly announcements to the class (e.g. every Monday via announcement forum, email)\\
7. Instructor created content in the form of short videos/multimedia (e.g., Camtasia, articulate modules)\\
8. Instructor being present in the discussion forums (e.g., refers to students by name, responds to students posts)\\
9. Instructor providing timely feedback on assignments/projects (e.g., within 7days)\\
10. Instructor providing feedback using various modalities (e.g., text, audio, video, and visuals) on assignments/projects\\
11. Instructors personal response to student reflections (e.g., via journals to questions on benefits/challenges\\
12. Instructors use of various features in synchronous sessions to interact with students (e.g., polls, emoticons, whiteboard, text, or audio and video chat)\\
13. Interactive visual syllabi of the course (e.g., includes visual of the instructor and other interactive components)\cite{Martin2018}\\
Online learning environments use different approaches to convey their information to their audience but when learning is most important thing in an online environment then we can use learning centered environment it uses number of factors to increase the learning rate. The design of online course materials is rarely informed by learning theories or their pedagogical implications. The goal of this research was to develop, implement and assess a virtual learning environment (VLE), SOFIAA, which was designed using the cognitive apprenticeship model (CAM), a pedagogical model based on learning-centered theory. We present an instructional design case study that reveals the steps taken to improve student performance in a master’s level blended learning course on program evaluation. The case study documents four phases of improving on-line instruction in program evaluation, starting with Online Course Materials (OCM) that contained resources and information required to complete team field projects. In phase 1, quantitative analyses revealed that there was improvement of student test scores using the OCM, however, qualitative analyses of think-aloud sessions found that students failed to attain key course objectives. In phase 2, a team of experts reviewed the materials and suggested ways to improve opportunities for student learning. In phase 3, a (VLE) was designed based on the results of phase 2 using a reconceptualization of CAM as a design model. In phase 4, the VLE was validated using experts’ appraisal of content and presentation, and student achievement, which indicated that use of the VLE led to significant improvement in learning over use of OCM. The design process is discussed in terms of a reconceptualization of CAM as a general strategy for instructional design that can be used to improve both the content and quality of online course materials.\cite{Garcia-Cabrero2018}\\
A study was conducted on the note taking quality of students in 2018. Many Students are terrible note takers who record just one third of a lecture's important points in their notes. This is not a good practice, because the number of lesson points recorded in notes is directly proportional to student's achievement. Moreover, both recording notes and the continuous review of notes are beneficial. The authors offer instructors a menu of research-based advice for increasing effectively student note taking: provide complete notes, partial notes, note- taking cues, represent the lesson, pauses and revision opportunities, control laptop usage, control "cyber-slacking", use slides effectively, and teach notes-taking skills to students. Authors also suggest ways to help students change their notes during the note-review process and select, organize, associate, and regulate their notes effectively to success. This study shows how notes taking helps student and how student are not good with it but we have to do something to improve it so their are some things or remedies we can do to improve it because notes are basically most essential thing for a student to get through the exams.\cite{Kiewra2018}\\
Engagement of students increase student's satisfaction, enhances student's motivation to learn, reduces the isolation sense, and improves performance of each student in online course. A survey-based research study was conducted to examine student's perception on different engagement strategies which are used in online courses based on  an interaction framework MOOR. 38 item survey was completed by one hundred and fifty-five students which was based on engagement strategies of learner-to-learner, learner-to-instructor and learner-to-content. The most valued engagement strategy was Learner-to-instructor among the three categories .In the learner-to-learner category the engagement strategy rated the most was Icebreaker/introduction discussions and working collaboratively using online communication, whereas in learner to instructor category sending regular announcements or email reminders and providing grading rubrics for all assignments were rated most beneficial. In the learner-content category, students mentioned working on real-world projects and having discussions with structured or guiding questions were the most beneficial. This study also analyzed the effects of age, gender and years of online learning experience differences on student's perception of engagement strategies. The results of the study have involved conclusion for online instructors, instructional designers, and administrators who wish to enhance engagement in the online courses. Basically, this paper shows how engagement matters in an online course because online courses can be a complete waste if we don't do it properly and study properly in it includes engagement of the student that he can follow the course as the instructor is teaching and what strategies we should use to engage a student in an online course. \cite{Martin2018(2)}\\
A study was conducted to analyze the behavior of students in massive open online courses (MOOCS) like how students show their confusion about things and how to improve that model so that student will have less confusion because more confusion creates less retention so if we can provide things in a way that students would have less confusion then they will focus on much more creativity and new ideas so in MOOCS we have to reduce the reasons of confusion as much as possible.\cite{Yang2016}\\
Discussion forums are also held in MOOCS and the odds of confusion is much more greater than the online courses because different people are giving different views and discussing them which can lead to confusion among the learners. The construction and destruction both are equally possible in discussion forums because confusion can change the discussion entirely and to prove that point a study was conducted in 2015 to examine the effects of confusion in MOOCS.\cite{Yang2015}\\
Then we have to see the patterns on which a student view and understand a video lecture for that cause a study was conducted in 2016 to see if their is any relationship between video lecture pattern and student perception of watching them and results shows that their were certain patterns on which a student watches an online lecture.\cite{Giannakos2016}\\
We have discussed all the pros and cons of video lectures and how to make them more efficient for the students but what if the quality or source of video becomes a problem video lectures takes a lot of space and a lot of speed is needed to stream them smoothly. If they are not streamed smoothly it is almost impossible to understand a lecture. A human mind loses interest in something after 2 seconds if its buffering and we feel angry and irritated so how can we focus on a video lecture which buffers after every 1 minute or even seconds. For that purpose different techniques were introduced to optimized the video size or quality to stream it smoothly. A good quality video is hard to stream but an average quality video can be streamed easily but what if we cannot compromise on quality. It is a seesaw but we have to balance both quality and space.\\
Automated video recording system was introduced in 2010 by a researcher in his paper. A PTZ camera was used in this study to capture the video lecture instead of a cameraman this camera acts as a cameraman and tracks the screen and lecturer and starts recording the lecture.\cite{Chou2010} 
The cons of this system was it provides the simple video and the streaming issue remains the same even if a student downloads every lecture he may need a lot of space to save them because 1 lecture of one hour can take space up to 1 GB.\\
Another technique to increase the quality and understanding of video lecture was introduced in 2017 by research team in their research paper. This technique captures video from different angles and multi-stream them. The approach was to use multiple cameras to get same scenes from different angles and stream the most suitable. The movements and actions of cameras were synchronized all the cameras captures the video at the same time. It increased the canvas of video lecture at that time later this technique was used in many gaming engines too. \cite{Carlsson2017} The con of this technique was that to capture a scene which is static in case of lecture video use of multiple cameras was expensive and quality wasn't even that much better.\\
A tracker named Gal Tracker was also introduced in 2013 by a researcher. The use of this tracker was to optimize the lecture recording. It captures the lecture in real time environment also called real time lecture capturing. It captures all the important parts of lecture and creates slides of lecture according which can be later distributed among the users.\cite{Gonzalez-Agulla2013} Again the issue is it can not capture all the important points given that it works on cues it can only capture a certain type of info and if something new comes it might fail in that environment. Its basically optimizing the lecture notes not lecture videos but trying to provide the same feature.\\
In MOOCS platform it is essential to provide videos but many people are working how to optimize them as much as possible and their type of optimization depends on how user can interact with video lectures. A research was conducted in 2015 to check how user interacts with single and multi streamed videos and their behavior on them. They also tested how it will help if a user can stream two videos at the same time.\cite{Renz2015} This statistical data from this research can be used to improve many new video streaming engines.\\
Since many people are streaming videos worldwide it is a wast field to compress these files but but compression can cause loss of data. To minimize the loss of data as much as possible different techniques of compression are used throughout the years. Wavelet transform is an efficient method to compress videos but in 2011 a research was conducted to achieve video compression algorithm using frame difference approach. This technique was to discard unimportant frames and keep the ones which can be important now the factors which decides whether the frame is important or not can be compression performance or frame change from previous one if its too minor it can be discarded. This technique was tested on many videos and the results were positive it can optimize the video size upto 3 percent and it's a big thing when we stream a 2 GB HD video with less amount of bandwidth.\cite{AlAni2011}
As we discussed earlier video compression is a wast field so here is another technique to compress a video: in 1992 an algorithm was developed to compress a video by MPEG video compression algorithm. The major focus of this algorithm is based on two techniques: block based motion compensation for the reduction of the temporal redundancy and transform domain based compression for the reduction of spatial redundancy. Motion compensation techniques are used with both interpolative and predictive techniques. The prediction error signal is further compressed with spatial redundancy reduction (DCT). This algorithm changed the video about 1.5Mbit/s.\cite{LeGall1992}\\
As we discussed different compression algorithms but they have some cons when we compress a video we compromise its quality a data loss is possible. There are some techniques which focuses on video compression as well as preserving the quality or resolution of video one of them was presented in 2016 which focuses on multi-level optimization of encoding to balance video compression and retention of 8k resolution. A paper was presented for this technique. This Paper is focused on providing better quality videos after compression by using some factors and techniques. This paper presents a technique that aimed to achieve an efficient balance between video compression using H.265 protocol and retention of 8K resolution. The study implements multi-level of optimization in the encoding process using H.265 where we can see JPEG2000 standards play an important role. The study also uses a novel concept of orthogonal projection that manages pixels metadata required in every frame transition by using motion compensation. By using multiple file formats of 30 video datasets, the outcome of the study is found to be achieving approximately 49 percent of increase in data quality and around 59 percent of enhancement in video compression in comparison to the former techniques of HEVC-based video compression which are most commonly used techniques.\cite{Murthy2016}\\
We saw how video lectures are helpful for learning and growth of students, how different video recording techniques are used to provide video lectures or notes, how compression of videos is important and how quality is important. We discussed different techniques and studies to achieve and justify all the goals. These techniques are not perfect but they are achieving the goal somehow "provide video lectures with ease". Ease can be considered as less space hogging and less bandwidth usage and still stream the lecture as possible. But first we have to see one thing to capture these lectures we have to change the environment of a classroom and we have to see either these changes are good or bad for classroom environment some studies were conducted to see the behavior of students towards these changes and the studies have shown positive result students were much attentive in lecture because they don't have to jot down the lecture they can focus on the concept because now they can access those lectures later on if wanted.\cite{Brooks2014}\cite{Chen2015}\cite{Barokas2010}\\
After discussing all the pros and cons of video lectures, availability of video lectures and techniques to make it easier. We are going to discuss another technique introduced by us. Compression may cause loss of data and quality may increase the space or bandwidth issue. To overcome this issue we are introducing DBM(digital board marker) this technique uses coordinate information to generate a video later on. DBM erases all the noise from video and chooses to save the information written on board or screen and audio of instructor. It draws a virtual plain on board and saves the coordinate information with respect to time in a customized file which can be later used to generate the video of board on our customized player integrated in web app and desktop application with a speed less than 2.5Kb. Which is much less than all the compression techniques used. So, DBM is solving all the compression problems for lecture videos and providing video lectures which is essential for the growth of student's learning.
 

 
\section{Comparison Table}

\begin{sideways}
\centering
\begin{tabularx}{1.5\textwidth} { 
  | >{\raggedright\arraybackslash}X 
  | >{\centering\arraybackslash}X | >{\centering\arraybackslash}X | >{\centering\arraybackslash}X | >{\centering\arraybackslash}X | }
 \hline
\bfseries{Name} & \bfseries{Support Video Lectures} &\bfseries{Provide Video Lectures} &\bfseries{Main Scope} & \bfseries{Video Optimization}  \\
\hline
Muti-level optimization in encoding to balance video compression and retention of 8k resolution.\cite{Murthy2016}
& Yes.
& Yes.
& H.265 encoding to balance compression and quality of video.
& 59 percent better compression and 49 percent increase in quality than ordinary compression.
\\
\hline
The use of lecture recording in higher education.\cite{OCallaghan2017}
& Yes
& Yes
& Simple video lectures captured by a simple camera.
& No
\\
\hline
Exploring the relationship between video lecture usage patterns and student attitudes.\cite{Giannakos2016}
& Yes
& Yes
& Sensors to analyze the usage pattern by students.
& Usage pattern can help to discard the unimportant information hence can optimize video.
\\
\hline
The impact of flipped classroom.\cite{Lo2019}
& Yes
& Yes
& Ordinary video lectures captured by a camera.
& No, ordinary video lectures.
\\
\hline
Anchoring interactive points of interest on web-based instruction video.\cite{Pimentel2019}
& Yes
& Yes
& Mark the important points on a web-based video.
& Yes, it can be used to optimize point of interest in video later.
\\
\hline
\end{tabularx}
\end{sideways}


\begin{sideways}
\centering
\begin{tabularx}{1.5\textwidth} { 
  | >{\raggedright\arraybackslash}X 
  | >{\centering\arraybackslash}X | >{\centering\arraybackslash}X | >{\centering\arraybackslash}X | >{\centering\arraybackslash}X | }
 \hline
\bfseries{Name} & \bfseries{Support Video Lectures} &\bfseries{Provide Video Lectures} &\bfseries{Main Scope} & \bfseries{Video Optimization}  \\
\hline
Modeling and quantifying the behaviors of students in lecture capture environments.\cite{Brooks2014}
& Yes
& Yes
& Analyze lecture capturing environment.
& No
\\
\hline
Student perception of helpfulness of facilitation strategies that enhance instructor presence, connectedness, engagement and learning in online courses.\cite{Martin2018}
& Yes
& Yes
& Provide helping material to students to analyze its effect.
& No
\\
\hline
Guided notes for university mathematics and their impact on students' notes-taking behavior.\cite{Iannone2019}
& No
& No
& Providing notes so that issue of notes-taking can be resolved.
& Better notes-taking can resolve the issue of capturing the complete video lecture.
\\
\hline



\end{tabularx}
\end{sideways}


\begin{sideways}
\centering
\begin{tabularx}{1.5\textwidth} { 
  | >{\raggedright\arraybackslash}X 
  | >{\centering\arraybackslash}X | >{\centering\arraybackslash}X | >{\centering\arraybackslash}X | >{\centering\arraybackslash}X | }
 \hline
\bfseries{Name} & \bfseries{Support Video Lectures} &\bfseries{Provide Video Lectures} &\bfseries{Main Scope} & \bfseries{Video Optimization}  \\
\hline
Effects of prior knowledge on attitudes, behavior and learning performance in video lecture viewing.\cite{Li2019}
& Yes.
& Yes.
& providing video lectures after the lecture and analyze its impact.
& No.
\\
\hline
Exploring the effects of confusion in discussion forums of massive open online courses.\cite{Yang2015}
& Yes.
& Yes.
& Identifying the confusion factors in online video lectures.
& No.
\\
\hline
Effects on learning time spent by university students attending lecture and watching online videos.\cite{Meehan2019}
& Yes.
& Yes.
& Providing ordinary video lectures and analyze the learning growth of student by time spent on them.
& No.
\\

\hline

\end{tabularx}
\end{sideways}


\begin{sideways}
\centering
\begin{tabularx}{1.5\textwidth} { 
  | >{\raggedright\arraybackslash}X 
  | >{\centering\arraybackslash}X | >{\centering\arraybackslash}X | >{\centering\arraybackslash}X | >{\centering\arraybackslash}X | }
 \hline
\bfseries{Name} & \bfseries{Support Video Lectures} &\bfseries{Provide Video Lectures} &\bfseries{Main Scope} & \bfseries{Video Optimization}  \\
\hline
Automated lecture recording system.\cite{Chou2010}
& Yes.
& Yes.
& Video lecture captured by PZT camera with motion and scene sensors.
& Yes, recording the video when it matches certain criterion.
\\
\hline
Design of learning-centered online environment.\cite{Garcia-Cabrero2018}
& Yes.
& Yes.
& Design an environment for video lectures with better learning.
& Yes, this design can optimize the video by capturing and providing only points of interest.
\\
\hline
Optimized adaptive streaming of multi-video stream bundles.\cite{Carlsson2017}
& Yes.
& Yes.
& Synchronized streaming of multiple videos.
& Yes, it uses encoding to optimize the streaming.
\\

\hline
GalTracker: Real time lecture tracking for lecture capturing.\cite{Gonzalez-Agulla2013}
& Yes.
& Yes.
& Provides slides by creating them in real time from frames of a video lecture.
& Yes, it discards the frames with minor distance and creates slides based on some factors.
\\
\hline

\end{tabularx}
\end{sideways}



\begin{sideways}
\centering
\begin{tabularx}{1.5\textwidth} { 
  | >{\raggedright\arraybackslash}X 
  | >{\centering\arraybackslash}X | >{\centering\arraybackslash}X | >{\centering\arraybackslash}X | >{\centering\arraybackslash}X | }
 \hline
\bfseries{Name} & \bfseries{Support Video Lectures} &\bfseries{Provide Video Lectures} &\bfseries{Main Scope} & \bfseries{Video Optimization}  \\
\hline
Lecture capture: student perceptions, expectations and behaviors.\cite{Barokas2010}
& Yes.
& Yes.
& Provide video lectures to analyze student behavior and expectations.
& No.
\\
\hline
Improving effectiveness of learners review of video lectures by using an attention based video lecture review mechanism  based on brainwave signals.\cite{Lin2019}
& Yes.
& Yes.
& Review video lectures based on brainwave signals of learner.
& Yes, video lecture can be optimized by the attention points of learner.
\\
\hline
Optimizing the video experience in MOOCS(massive open online courses).\cite{Renz2015}
& Yes.
& Yes.
& Increase the streaming quality in MOOCS.
& Yes, it optimizes the video by providing a timeline with video lecture.
\\

\hline
\end{tabularx}
\end{sideways}



\begin{sideways}
\centering
\begin{tabularx}{1.5\textwidth} { 
  | >{\raggedright\arraybackslash}X 
  | >{\centering\arraybackslash}X | >{\centering\arraybackslash}X | >{\centering\arraybackslash}X | >{\centering\arraybackslash}X | }
 \hline
\bfseries{Name} & \bfseries{Support Video Lectures} &\bfseries{Provide Video Lectures} &\bfseries{Main Scope} & \bfseries{Video Optimization}  \\
\hline
Exploring the effects of students confusion in massive open online courses.\cite{Yang2016}
& Yes.
& Yes.
& Identify the problems because of confusion in MOOCS.
& No.
\\
\hline
Engagement Matters: student perception on the importance of engagement strategies in the online learning environment.\cite{Martin2018(2)}
& No.
& No. 
& Analyzing the student views on engagement with instructor for learning.
& No.
\\

\hline
Video Compression Algorithm based on Frame Difference Approaches.\cite{AlAni2011}
& Yes.
& Yes.
& Video compression based on difference in frames by some factors: performance and difference from previous frame.
& Yes, it optimizes the video size by 3 percent.
\\
\hline

\end{tabularx}
\end{sideways}


\begin{sideways}
\centering


\begin{tabularx}{1.5\textwidth} { 
  | >{\raggedright\arraybackslash}X 
  | >{\centering\arraybackslash}X | >{\centering\arraybackslash}X | >{\centering\arraybackslash}X | >{\centering\arraybackslash}X | }  
  
 \hline
\bfseries{Name} & \bfseries{Support Video Lectures} &\bfseries{Provide Video Lectures} &\bfseries{Main Scope} & \bfseries{Video Optimization}  \\
\hline
The MPEG Video Compression Algorithm.\cite{LeGall1992}
& Yes.
& Yes.
& Video compression by MPEG algorithm by interpolation and predictive technique.  
& Yes, it optimizes video by 1.5 Mbits
\\
\hline

Recorded lectures as a source of cognitive off loading.\cite{Patel2019}
& Yes.
& Yes.
& Record video lectures and provide them to students to increase their learning growth.
& No.
\\

\hline
How to take good notes.\cite{Kiewra2018}
& No.
& No.
& Increase student learning growth by taking good notes and review them later.
& No.
\\
\hline

%\caption{Web App: Test Scenario TS-1 Results}
\end{tabularx}

\end{sideways}


\section{Shortcomings in Existing Systems}
Like all the other systems existing systems also have some shortcomings which may or may not be overcome by new systems such as:
\begin{itemize}
	\item Students can become lazy.
    \item Attendance rate can decrease because of availability of all the resources.
    \item Student behavior can change which can change the premise of lecture.
    \item Using ordinary video capture techniques might not be suitable for all the students.
    \item Students who don't or cannot have suitable internet connection cannot stream the video lectures.
    \item Some students cannot have the space to store ordinary lecture videos because of their size.
    \item Compression might cause loss in data.
    \item No matter how accurately we can use the factors to compress a video but we are discarding some frames that may cause loss of crucial information.
    \item Quality of video is also compromised in some systems which is not tolerated in some cases.
    \item It can be difficult to adapt the new environment for both students and instructor.
    \item Instructor will be a little conscious in new lecture capturing environment.
    \item Students can become non-serious about taking lectures in real time environment so we have to make sure the attendance is compulsory.
    \item A platform has to be introduced for video lectures availability.
    \item Maintaining those video lectures can be difficult. 
\end{itemize}



